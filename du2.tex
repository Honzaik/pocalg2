\documentclass[12pt, a4paper]{article}
\usepackage[margin=1in]{geometry}
\usepackage[utf8x]{inputenc}
\usepackage{indentfirst} %indentace prvního odstavce
\usepackage{mathtools}
\usepackage{amsfonts}
\usepackage{amsmath}
\usepackage{amssymb}
\usepackage{graphicx}
\usepackage{enumitem}
\usepackage{subfig}
\usepackage{float}
\usepackage[czech]{babel}
\usepackage{mathdots}
\usepackage{slashbox}

\begin{document}
Cvičení 8:\\

Definujeme si souřadnicový systém tak, aby body $A, B, C$ měly tyto souřadnice $A = (0,0), B = (2S_x, 0), C = (C_x, C_y)$. První souřadnici nazýváme $x$ souřadnici a druhou jako $y$. 

Bod $S = (S_x,S_y)$ je střed opsané kružnice. Z definice je $S$ stejně vzdálený od bodu $A$ jako od bodu $B$, tudíž můžeme souřadnice bodu $B$ definovat pomocí hodnoty $S_x$. Budeme predpokládat, že $S_x \neq 0$, aby nenastalo $A=B$.

Bod $C$ už nelze více zjednodušit, proto má $2$ obecné souřadnice.

Body $P, L, M$ jsou také obecné a proto $P = (P_x, P_y), L = (L_x, L_y), M = (M_x, M_y)$. Ohledně bodu $K$ můžeme říci, že jeho druhá souřadnice je $0$ (protože leží na přímce mezi $A,B$), proto $K = (K_x, 0)$.

Zadání nám dává následující podmínky, které můžeme zapsat pomocí rovnic s výše uvedenými proměnými a pomocnými $z_i$:
\begin{align}
&S_x \neq 0 \iff S_x z_1 - 1 = 0\\
&\Vert \overrightarrow{PS} \Vert = \Vert \overrightarrow{AS} \Vert \iff P_x^2 - 2(P_xS_x + P_yS_y) + P_y^2 = 0\\
&\Vert \overrightarrow{AS} \Vert = \Vert \overrightarrow{CS} \Vert \iff C_x^2 - 2(C_xS_x + C_yS_y) + C_y^2 = 0\\
& C_x \neq 0 \land C_y \neq 0 \iff (C_x^2 +C_y^2)z_2 - 1 = 0\\ 
&\overrightarrow{PK} \perp \overrightarrow{AB} \iff 
\begin{pmatrix}
P_x - K_x \\
P_y
\end{pmatrix} \cdot  
\begin{pmatrix}
-2S_x \\
0
\end{pmatrix} = 0 \stackrel{S_x \neq 0}{\iff} K_x - P_x = 0\\
&\overrightarrow{PL} \perp \overrightarrow{BC} \iff 2P_xS_x - P_xC_x - 2L_xS_x + L_xC_x - C_yP_y + C_yL_y = 0\\
&\overrightarrow{PM} \perp \overrightarrow{AC} \iff C_xP_x - C_xM_x + C_yP_y - C_yM_y = 0\\
&P_y \neq 0 (\implies P \neq A \land P \neq B) \iff P_y z_3 - 1 = 0\\
&P \neq C \iff ((P_x - C_x)^2 + (P_y - C_y)^2)z_4 - 1 = 0
\end{align}

Máme 9 rovnic v 16 proměnných popisující předpoklady. Nyní potřebujeme ještě rovnici popisující, že $K,L,M$ jsou na stejné přímce. Pro to stačí například, že $M$ leží na přímce $\overrightarrow{LK}$. Rovnici přímky $\overrightarrow{LK}$ má tvar $L_y x + (K_x - L_x)y - L_yK_x = 0$. Chceme, aby $M$ splňoval tuto rovnici. Po dosazení dostaneme rovnici 
\[L_yM_x + K_xM_y - L_xM_y - L_yK_x = 0\]

Označme polynomy z pravých stranr rovnic výše symboly $f_1, \dots, f_9$ a poslední polynom \uv{přímky} písmenem $g$. Z přednášky víme, že pro důkaz potřebujeme, aby $g \in I(V(f_1, \dots, f_9)) \iff 1 \in (f_1, \dots, f_9, gt-1)$. Náležení do radikálu odpovídá tomu, že všechny body, které splňují předpoklady splňují i důsledek neboli dokazují větu. Stačí tedy spočítat normovanou Groebnerovu bázi ideálu $(f_1, \dots, f_9, gt-1)$ a zkontrolovat zda vyjde 1. V našem případě tyto rovnice postačují.
\end{document}

