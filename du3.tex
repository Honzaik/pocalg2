\documentclass[12pt, a4paper]{article}
\usepackage[margin=1in]{geometry}
\usepackage[utf8x]{inputenc}
\usepackage{indentfirst} %indentace prvního odstavce
\usepackage{mathtools}
\usepackage{amsfonts}
\usepackage{amsmath}
\usepackage{amssymb}
\usepackage{graphicx}
\usepackage{enumitem}
\usepackage{subfig}
\usepackage{float}
\usepackage[czech]{babel}
\usepackage{mathdots}
\usepackage{slashbox}

\begin{document}

\section{}
Provedeme GS ortogonalizaci na vektorech $b_1, b_2, b_3$. Dostaneme $b_1^* = (0, 3, 4), b_2^* = (-1, \frac{12}{25}, \frac{-9}{25}), b_3^* = (\frac{78}{17}, \frac{104}{17}, \frac{-78}{17})$ a hodnoty $\mu_{2,1} = \frac{21}{25}, \mu_{3,1} = \frac{-16}{25}, \mu_{3,2} = \frac{-7}{17}$. Nyní můžeme přejít k druhému kroku LLL:
\begin{align*}
&i=2, j=1:\\
&x \coloneqq 1\\
&b_2 \coloneqq b_2-1b_1=\begin{pmatrix}
-1\\
0\\
-1
\end{pmatrix}\\
&\mu_{2,1} \coloneqq \mu_{2,1}-1 = \frac{-4}{25}\\\\
&i=3, j=2:\\
&x \coloneqq 0 \text{ (tedy zbytek tohoto kroku můžeme přeskočit)}\\\\
&i=3, j=1:\\
&x \coloneqq -1\\
&b_3 \coloneqq b_3 - (-1)b_1 = \begin{pmatrix}
5\\
7\\
-3
\end{pmatrix}\\
&\mu_{3,1} = \mu{3,1} - (-1) = \frac{9}{25}
\end{align*}

Nyní můžeme přejít ke \uv{kontrolnímu} kroku LLL. Zjistíme, ale že musíme prohodit vektory $b_1$ a $b_2$, protože $\| b_2^* \|^2 < (\frac{3}{4} - \mu_{2,1}^2)\| b_1^* \|^2 \iff \frac{34}{25} < (\frac{3}{4}-(\frac{-4}{25})^2)25$. 

Nově máme tedy prohozené vektory $b_1 = (-1, 0, -1)^T, b_2 = (0, 3, 4)^T, b_3 = (5, 7, -3)^T$. Spočítáme opět GS ortogonalizaci: $b_1^* = (-1, 0, -1)^T, b_2^* = (-2, 3, 2)^T, b_3^* = (\frac{78}{17}, \frac{104}{17}, \frac{-78}{17})^T$, $\mu_{2,1} = -2, \mu_{3,1} = -1, \mu_{3,2} = \frac{5}{17}$. Druhý krok:
\begin{align*}
&i=2, j=1:\\
&x \coloneqq -2\\
&b_2 \coloneqq  b_2 - (-2)b_1 = \begin{pmatrix}
-2\\
3\\
2
\end{pmatrix}\\
&\mu_{2,1} \coloneqq -2 - (-2) = 0\\\\
&i=3, j=2:\\
&x \coloneqq 0\\\\
&i=3, j=1:\\
&x \coloneqq -1\\
&b_3 \coloneqq b_3 - (-1)b_1 = \begin{pmatrix}
4\\
7\\
-4
\end{pmatrix}
&\mu_{3,1} = \mu_{3,1} - (-1) = 0
\end{align*}

Kontrolou tentokrát projdou všechny vektory. Výstupem LLL jsou tedy vektory:
\[
b_1 = \begin{pmatrix}
-1\\
0\\
-1
\end{pmatrix},
b_2 = \begin{pmatrix}
-2\\
3\\
2
\end{pmatrix}
b_3 = \begin{pmatrix}
4\\
7\\
-4
\end{pmatrix}
\]
\end{document}

